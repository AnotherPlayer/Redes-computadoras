\documentclass{article}
\usepackage{geometry}
\usepackage{enumitem}
\usepackage{graphicx}
\usepackage{ragged2e}
\usepackage{listings}
\usepackage{xcolor}

\lstset{
    language=C, basicstyle=\ttfamily\small, keywordstyle=\color{blue},
    stringstyle=\color{red}, commentstyle=\color{gray}, frame=single,
    breaklines=true, showstringspaces=false
}

\begin{document}

\begin{titlepage}

    \flushleft
    \includegraphics[width=0.3\textwidth]{./../../escom.png} 
    \vspace{2cm}\\
    {\bfseries\LARGE Escuelas Supeior de Cómputo \\}
    \vspace{3cm}
    {\scshape\Huge Práctica 3 \\ Recepción de datos a través de un socket no orientados a conexión (UDP)\\}
    \vspace{3cm}
    {\itshape\Large Redes de computadora \\}
    \vfill
    {\Large Autor: \\}
    {\Large Héctor David González Tetuán \\}
    \vfill
    {\Large 15/11/2025 \\}

\end{titlepage}

\newpage

\section*{Código en lenguaje C del programa para recibir un mensaje a través de un socket UDP.}

\begin{lstlisting}
    
#include <stdio.h>
#include <stdlib.h>
#include <string.h>
#include <unistd.h>
#include <sys/socket.h>
#include <netinet/in.h>
#include <arpa/inet.h>

int main() {

    int udp_socket, lbind, tam, lrecv;
    struct sockaddr_in servidor, cliente;
    unsigned char msj[512];
    unsigned char paqRec[512];
    socklen_t len_cliente;

    udp_socket = socket(AF_INET, SOCK_DGRAM, 0);

    if(udp_socket == -1){
        perror("Error al abrir el socket");
        exit(EXIT_FAILURE);
    }
    else{

        perror("Exito al abrir el socket");
    
        memset(&servidor, 0, sizeof(servidor));
    
        servidor.sin_family = AF_INET;
        servidor.sin_port = htons(8080);
        servidor.sin_addr.s_addr = INADDR_ANY;

        lbind = bind(udp_socket, (struct sockaddr *)&servidor, sizeof(servidor));

        if(lbind == -1){
            perror("Error en bind");
            exit(EXIT_FAILURE);
        }
        else{

            perror("Exito en bind");

            while (1){
            
                lrecv = sizeof(cliente);
                tam = recvfrom(udp_socket, paqRec, 512, 0, (struct sockaddr *)&cliente, &lrecv);

                if(tam == -1){
                    perror("Error al recibir");
                    exit(0);
                }
                else 
                    printf("\nEl mensaje recibido es: %s\n", paqRec);

            }
        }
    }

    close(udp_socket);
    return 0;

}

\end{lstlisting}

\newpage

\section*{Captura de pantalla de la terminal donde se muestra que recibió un mensaje a través de un socket UDP.}

\includegraphics[width=0.9\textwidth]{./img/getMensaje.png}

\newpage

\section*{Código en lenguaje C del programa cliente chat.}

\begin{lstlisting}
    
#include <stdio.h>
#include <stdlib.h>
#include <string.h>
#include <unistd.h>
#include <sys/socket.h>
#include <netinet/in.h>
#include <arpa/inet.h>

void writeText( char* msj ){

    printf("\nEscribe un mensaje (Usuario): \n");
    fgets(msj, 512, stdin);
    msj[strcspn(msj, "\n")] = 0;

}

int main() {

    int udp_socket, lbind, tam, lrecv;
    struct sockaddr_in local, remota;
    unsigned char msj[512];
    unsigned char paqRec[512];
    udp_socket = socket(AF_INET, SOCK_DGRAM, 0);

    if(udp_socket == -1){
        perror("Error al abrir el socket");
        exit(EXIT_FAILURE);
    }
    else{

        perror("Exito al abrir el socket");
    
        local.sin_family = AF_INET;
        local.sin_port = htons(0); 
        local.sin_addr.s_addr = INADDR_ANY;

        lbind = bind(udp_socket, (struct sockaddr *)&local, sizeof(local));

        if(lbind == -1){
            perror("Error en bind");
            exit(0);
        }
        else{

            perror("Exito en bind");

            remota.sin_family = AF_INET;
            remota.sin_port = htons(8080); // Puerto DNS
            remota.sin_addr.s_addr = inet_addr("10.100.82.16");

            while (1){
                
                writeText(msj);

                tam = sendto(udp_socket, msj, 512, 0, (struct sockaddr *)&remota, sizeof(remota));
                
                if(tam == -1){
                    perror("Error al enviar");
                    exit(0);
                }
                else
                    printf("\nEnviando mensaje a Servidor: %s\n", msj);

                lrecv = sizeof(remota);
                tam = recvfrom(udp_socket, paqRec, 512, 0, (struct sockaddr *)&remota, &lrecv);

                if(tam == -1){
                    perror("Error al recibir");
                    exit(0);
                }
                else 
                    printf("\nEl mensaje recibido es: %s\n", paqRec);

            }

        }

    }
                
    close(udp_socket);
    return 0;

}

\end{lstlisting}

\newpage

\section*{Código en lenguaje C del programa servidor chat.}

\begin{lstlisting}

#include <stdio.h>
#include <stdlib.h>
#include <string.h>
#include <unistd.h>
#include <sys/socket.h>
#include <netinet/in.h>
#include <arpa/inet.h>

void writeText( char* msj ){

    printf("\nEscribe un mensaje (Servidor): \n");
    fgets(msj, 512, stdin);
    msj[strcspn(msj, "\n")] = 0;

}

int main() {

    int udp_socket, lbind, tam, lrecv;
    struct sockaddr_in servidor, cliente;
    unsigned char msj[512];
    unsigned char paqRec[512];
    socklen_t len_cliente;

    udp_socket = socket(AF_INET, SOCK_DGRAM, 0);

    if(udp_socket == -1){
        perror("Error al abrir el socket");
        exit(EXIT_FAILURE);
    }
    else{

        perror("Exito al abrir el socket");
    
        memset(&servidor, 0, sizeof(servidor));
    
        servidor.sin_family = AF_INET;
        servidor.sin_port = htons(8080);
        servidor.sin_addr.s_addr = INADDR_ANY;

        lbind = bind(udp_socket, (struct sockaddr *)&servidor, sizeof(servidor));

        if(lbind == -1){
            perror("Error en bind");
            exit(EXIT_FAILURE);
        }
        else{

            perror("Exito en bind");

            while (1){
            
                lrecv = sizeof(cliente);
                tam = recvfrom(udp_socket, paqRec, 512, 0, (struct sockaddr *)&cliente, &lrecv);

                if(tam == -1){
                    perror("Error al recibir");
                    exit(0);
                }
                else 
                    printf("\nEl mensaje recibido es: %s\n", paqRec);

                writeText(msj);
            
                tam = sendto(udp_socket, msj, 512, 0, (struct sockaddr *)&cliente, sizeof(cliente));
            
                if(tam == -1){
                    perror("Error al enviar");
                    exit(0);
                }
                else
                    printf("\nEnviando mensaje a Usuario: %s\n", msj);

            }
        }
    }

    close(udp_socket);
    return 0;

}

\end{lstlisting}

\newpage

\section*{Capturas de pantalla donde se muestra la interacción de envío de mensajes entre cliente y servidor.}

\includegraphics[width=0.99\textwidth]{./img/chat.png}

\newpage


\end{document}