\documentclass{article}
\usepackage{geometry}
\usepackage{enumitem}
\usepackage{graphicx}
\usepackage{ragged2e}

\begin{document}

\begin{titlepage}

    \flushleft
    \includegraphics[width=0.3\textwidth]{./../../escom.png} 
    \vspace{2cm}\\
    {\bfseries\LARGE Escuelas Supeior de Cómputo \\}
    \vspace{3cm}
    {\scshape\Huge Práctica 1 \\ Fundamentos de Redes de Computadoras \\}
    \vspace{3cm}
    {\itshape\Large Redes de computadora \\}
    \vfill
    {\Large Autor: \\}
    {\Large Héctor David González Tetuán \\}
    \vfill
    {\Large 27/10/2025 \\}

\end{titlepage}

\newpage

\section*{Cuestionario}
\begin{enumerate}[label=\arabic*.]

    \item ¿Qué es el modelo OSI en el contexto de redes de computadoras? \\[0.3cm] 
    El modelo de Interconexión de Sistemas Abiertos (OSI), es un modelo conceptual creado por la Organización Internacional para la Estandarización
    que divide la comunicación y la interoperabilidad de la red en siete capas abstractas.\\
    En otras palabras, es un estandar para que distintos sistemas puedan comunicarse.
    \\[0.3cm]
    
    \item ¿Qué topologías de red utiliza un concentrador central para conectar todos los dispositivos? \\[0.3cm]
    La topología utlizada para esta conexión de la de tipo estrella, ya que con esta configuración cada dispositivo tiene un enlace punto a punto
    que va hacia el centro.
    \\[0.3cm]
    
    \item ¿Qué protocolo se utiliza comúnmente para la transferencia de archivos en una red? \\[0.3cm]
    El protocolo FTP (Protocolo de transferencia de archivos) es un protocolo de réd para estandarizar la
    transferencia de archivos de un host a otro a través de una red en TC.
    \\[0.3cm]
    
    \item ¿Qué tipo de red se utiliza para conectar dispositivos en un área geográfica limitada, como una casa u oficina? \\[0.3cm]
    Una red LAN (Local Area Network).
    \\[0.3cm]
    
    \item ¿Qué función principal realiza un router en una red? \\
    Encamina o dirige los paquetes de datos entre diferentes redes, determinando la mejor ruta para llegar al destino.
    \\[0.3cm]
    
    \item ¿Qué funciones describen la calidad de servicio (QoS) en una red? \\
    Priorizar el tráfico de red, controlar el ancho de banda y garantizar un rendimiento óptimo para aplicaciones críticas (como voz o video en tiempo real).
    \\[0.3cm]
    
    \item ¿Qué protocolo se utiliza para traducir nombres de dominio en direcciones IP? \\
    El DNS (Domain Name System).
    \\[0.3cm]
    
    \item ¿Qué tecnología se utiliza para permitir a los empleados acceder de forma segura a la red de la empresa desde ubicaciones remotas? \\
    Una VPN (Virtual Private Network).
    \\[0.3cm]
    
    \item ¿Qué funciones se realizan para desarrollar la administración de redes? \\
    Monitorear, configurar, optimizar, asegurar y mantener los dispositivos y servicios de red para garantizar su correcto funcionamiento.
    \\[0.3cm]
    
    \item ¿Qué tecnologías de red se utilizan para crear redes virtuales aisladas dentro de una red física? \\
    Las VLAN (Virtual Local Area Network).
    \\[0.3cm]

    \item ¿Qué función realiza un firewall en una red de computadoras? \\
    Controla y filtra el tráfico de red entrante y saliente, protegiendo la red contra accesos no autorizados y ataques.
    \\[0.3cm]
    
    \item ¿Qué tecnología permite a los empleados acceder de forma segura a la red de la empresa desde ubicaciones remotas a través de una conexión cifrada? \\
    También una VPN (Virtual Private Network).
    \\[0.3cm]
    
    \item ¿Cuál es el propósito principal de la gestión de redes en una organización? \\
    Garantizar que la red funcione de manera eficiente, segura y disponible, manteniendo el rendimiento y minimizando fallos.
    \\[0.3cm]
    
    \item ¿Qué protocolo se utiliza para asignar dinámicamente direcciones IP a dispositivos en una red? \\
    El DHCP (Dynamic Host Configuration Protocol).
    \\[0.3cm]
    
    \item ¿Cuál de las siguientes tecnologías de red se utiliza para priorizar el tráfico de red y garantizar un rendimiento óptimo para aplicaciones críticas? \\
    La QoS (Quality of Service).
    \\[0.3cm]
    
    \item ¿Qué capa del modelo OSI se encarga de la segmentación y reensamblaje de datos en paquetes? \\
    La Capa de Transporte (Capa 4).
    \\[0.3cm]

\end{enumerate}

\newpage

\section*{Conclusión}

El cuestionario permitió reforzar los conocimientos fundamentales sobre redes de computadoras, comprendiendo cómo se estructuran, comunican y protegen los sistemas dentro de un entorno conectado.
Conocer estos fundamentos es necesario para el diseño, administración y mantenimiento de infraestructuras de red confiables que soporten la comunicación y el intercambio de información en organizaciones modernas.

\end{document}